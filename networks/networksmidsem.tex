\documentclass[a4paper,10pt]{article}
\usepackage{amsmath}
\usepackage[utf8]{inputenc}

%opening
\title{Networks ES 404}
\author{Mid Semester, Spring 2014}

\begin{document}

\maketitle
\begin{center}
 \textbf{Total: 100 points (+ 4 bonus points). Time: 2.5 hours. Please justify your answers whenever suitable. Please also show your work.} 
 \line(1,0){300}
\end{center}


\begin{enumerate}
 \item Consider the following networks:
 \vspace*{2in}
 
 
 Network (a) is directed, network (b) is undirected and bipartite. Write down:
 \begin{enumerate}
 \item (5 points) The adjacency matrix of (a).
 \item (5 points) The cocitation matrix of (b).
 \item (5 points) The indicence matrix of (a).
 \item (5 points) The projection matrix of (b) onto its black vertices. 
 \item (5 points) The bibliographic coupling matrix of (a). 
\end{enumerate}


\item 
\begin{enumerate}
\item (2 points) For a graph $G$, define $k_{\max}(G)$ to be the maximum $k$ such that the $k$-core of $G$ is non-empty.
Give upper and lower bound for $k$ in terms of the degrees (e.g. maximum or minimum) of $G$. 
\item (5 points) For a graph $G$, and a number $k$, how would you find out a set that is a $k$-core? No need to prove
the correctness of your algorithm.
\item (5 points) Give an example of a network which has a 3-core but no 4-core. Justify your answer.
\end{enumerate}


\item 
\begin{enumerate}
\item (2 points) Define closeness centrality.
\item (6 points) Consider an undirected tree of $n$ vertices. A particular edge in the tree joins vertices $1$ and $2$
and divides the tee into disjoint regions of $n_1$ and $n_2$ vertices, as sketched here. 
\vspace*{2in}

Let $C_1$ be the closeness
centrality of $1$ and $C_2$ be the closeness centrality of $2$. Show that
\begin{align*}
\frac{1}{C_1} + \frac{n_1}{n} = \frac{1}{C_2} + \frac{n_2}{n}
\end{align*}
(Hint: For a vertex in the left hand side, how does the distance of the vertex to $1$ relate to its distance to $2$? 
Use this information to find out the relation between $C_1$ and $C_2$.) 

\end{enumerate}

\item 
\begin{enumerate}
\item (2 points) Define betweenness centrality of a vertex. 
\item (6 points) Consider an undirected connected tree of $n$ vertices. 
Suppose that a particular vertex in the tree has degree $k$, so that if we remove that vertex, we get $k$ disjoint parts.
Suppose the size of the $k$ parts is $n_1$, $n_2$,...,$n_k$. Show that the betweenness centrality $x$ of the vertex
is 
\begin{align*}
 x = n^2 - \sum_{i=1}^k n_i^2.
\end{align*}
(Hint: Recall that in a tree there is only path between pairs of edges.)
\item (4 points) Hence, or otherwise, calculate the betweenness centrality of the $i^{th}$ vertex from the end of a line-graph
of $n$ vertices, i.e. when $n$ vertices are arranged in a row as follows:
\vspace*{1.5in}
\end{enumerate}


\item Suppose a graph on $n$ nodes has the following degree distribution. 
\begin{align*}
 p_k &= \mbox{ fraction of vertices of degree } k\\
\end{align*}
Then, there is a constant $C$ and a constant $\alpha> 1$ such that
\begin{align*}
 p_k = \frac{C}{k^\alpha} \mbox{ for } k \ge 1.
\end{align*}
Recall that $p_k$ is a density function, hence $\sum_{k=1}^{\infty} p_k = 1$. 
\begin{enumerate}
\item (5 points) Using the above relation give an approximate value for $C$ in terms of the constant $\alpha$.
\item (2 points) Give a bound on the maximum degree in the graph (remember that the number of nodes is $n$).
\item (5 points) What is the average degree of the graph?
\end{enumerate}

\item Consider a graph $G$ generated from the Erdos-Renyi random graph model, 
i.e. for every pair of vertices $(i, j)$, the 
edge $E_{ij}$ is a random variable that is given by 
\begin{align*}
  E_{ij} = \begin{cases}
           1 \mbox{ with probability } p;\\
           0 \mbox{ with probability } 1 - p.
            \end{cases}
\end{align*}
\begin{enumerate}
 \item (5 points) Compute the expected degree of the graph $G$? Justify your answer.
 \item (5 points) Compute the probability that a vertex has degree exactly $k$ ? Again, justify your answer. 
 \item (5 points) Recall that local clustering coefficient of a node $v$ is defined as the probability that two neighbors of a node $v$ are connected. 
 Using this definition, what is the clustering coefficient of any node? What is then the average local clustering coefficient?
\end{enumerate}





\item 
\begin{enumerate}
\item (2 points) What is the diameter of a clique?
\item (5 points) What is the diameter of a square lattice with $L$ edges and $L+1$ vertices in each side? The following example has $L=5$.
Give the answer for a general $L$.
\vspace*{2in}

\item A (5 points) Cayley tree is a symmetric regular tree in which each vertex is connected to the same number $k$ of vertices, 
until we get to the leaves. The following example shows a Cayley tree with $k = 3$. 
\vspace*{2in}

Show that the number of vertices reachable in $d$ steps from the central vertex is $k(k-1)^d$ for $d \ge 1$. 
Hence find an expression for the diameter of the Cayley tree. 

\item (2 points) Which of the networks in (a), (b) or (c) behave like a ``small world network'', as defined by having a diameter
that increases as $\log(n)$ or slower?
\end{enumerate}

\item Let $G$ be a $k$ regular undirected graph (i.e. a network in which every vertex has degree $k$).
\begin{enumerate}
\item (3 points) Show that the vector $\mathbf{1} = (1, 1, \ldots, 1)$ is an eigenvector of the adjacency matrix of $G$. Recall that
an a vector is an eigenvector of a matrix $A$ if $Av = \lambda v$ for some $\lambda$. 
\item (2 points) What are the eigenvector centralities of the vertices of $G$?
\item (2 points) Name a centrality measure that would take different values on the nodes of $G$. 
\item (Bonus 4 points) Give an example of a regular graph $G$ where for two nodes in $G$, your centrality measure has different values. 
\end{enumerate}




\end{enumerate}

\end{document}
 A network consists of a ring of $n$ vertices, each of which is connected to the $r$ vertices immediately.
clockwise from it around the ring and to the $r$ vertices immediately anticlockwise. Thus each
vertex has degree $k = 2r$. You can assume that $n >>r$.
\begin{enumerate}
\item How many “connected triples” are there in the network, i.e., a vertex plus an
unordered pair of its neighbors?
\item How many triangles are there in the network, i.e., trios of vertices each connected
to both of the others?
\item What is the clustering coeffcient of the network?
\end{enumerate}
(\textbf{Hint}: Do a small example with $r = 5$. Draw a vertex with $5$ vertices to its right and 
